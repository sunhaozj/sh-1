\vspace{-2.5cm}
\chapter*{\zihao{2}\heiti{摘~~~~要}}
\vspace{-1cm}
智能终端的普及推动了移动互联网进入一个高度发展的时代。GPS定位技术以及无线通信的日益完善也给人类的衣食住行提供了极大的便利,如今交通、医疗、教育等这些和民众生活息息相关的服务都离不开智能终端、无线通信等技术的支持。基于位置的服务(LBS;Location-Based Service)是地理位置和移动互联网结合,是当下信息化时代比较耀眼的模式之一。利用LBS,用户可以获得当前位置下的兴趣点(餐厅、KTV、影院...),但同时也暴露出一些隐私问题。用户和LBS服务提供商提出查询服务的时候,首先需要通过GPS获取到自己当前的位置,然后通过基站将位置发送给LBS服务提供商。由于用户的位置也许会揭露出一些敏感个人信息,而用户的当前位置完全暴露在LBS服务提供商面前,因此对用户的位置隐私构成了威胁。

本文主要工作包括以下几个方面:
\begin{itemize}
  \item \textbf{位置隐私保护方法的对比分析} ~~提出了海量轨迹数据的分布式处理框架,分别讨论通用轨迹数据处理中的噪声过滤、路网匹配和特征抽取三个阶段利用Map-Reduce的计算方案,并实现了本文的路网匹配RouteFit 算法。
  \item \textbf{针对KNN查询的隐私保护算法} ~~提出了海量轨迹数据的分布式处理框架,分别讨论通用轨迹数据处理中的噪声过滤、路网匹配和特征抽取三个阶段利用Map-Reduce的计算方案,并实现了本文的路网匹配RouteFit 算法。
  \item \textbf{时空数据聚合隐私保护算法} ~~采用基于密度的聚类方法来发现位置点数据中的兴趣点和兴趣区域,通过实现Pick-up DBScan算法来完成对出租车轨迹数据中具有语义特征的上下客位置点的聚类,生成候选出租车扬招POI和热门目的地ROI,为推荐提供重要数据集。
  \item \textbf{基于差分隐私的位置隐私保护算法} ~~介绍了利用海量出租车轨迹数据来优化出行的位置推荐服务,提出了出租扬招位置查询和候车时间预测系统,以推荐合理的出租车扬招位置点和预测准确的候车时间为目的,离线处理部分通过分布式轨迹处理框架完成轨迹预处理和特征抽取工作,以路段聚类的方法来划分模型粒度,设计多种空车等候时间的预测模型并进行评估和选择,在线查询部分利用空间索引技术和Web 服务技术实现对输入查询点的实时位置推荐服务,最后实现了基于上海市区大规模出租车轨迹历史数据的处理和分析预测的原型系统,提供对出租车扬招点得位置推荐服务。
\end{itemize}
\hspace{-0.5cm}
\sihao{\heiti{关键词:}} \xiaosi{移动推荐,LBS,轨迹挖掘}
